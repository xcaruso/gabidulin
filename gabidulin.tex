\documentclass[a4paper]{llncs}

\usepackage[utf8]{inputenc}
\usepackage{amsmath}
\usepackage{amsfonts}
\usepackage{amssymb}
\usepackage{color}
\usepackage[english]{babel}
\usepackage{microtype}

\definecolor{todo}{rgb}{0.9,0,0}

\author{Caruso Xavier \and Durand Amaury}
\institute{Institute : to do}

\def\todo#1{{\color{todo} #1}}

\title{Gabidulin Codes}

\newcommand{\ZZ}{\mathbb Z}

\newcommand{\id}{\textrm{id}}
\newcommand{\End}{\textrm{End}}
\newcommand{\GL}{\textrm{GL}}
\newcommand{\ev}[1]{\textrm{ev}_{#1}}
\renewcommand{\mod}{\,\%\,}

\begin{document}
\maketitle

\section{Introduction}

In 1985, Gabidulin Ernst introduced a Reed-Solomon-like code 
construction in rank metric. This Gabidulin code class use linearised 
polynomial concept. In 2013, Wachter-Zeh Antonia proposed efficient 
implementations of operations with linearized polynomials as well as an 
equivalent of Gao's algorithm for decoding Gabidulin code. Following 
years, Boucher Delphine and Ulmer Felix have worked on skew coding 
theory using Ore's polynomials and they have generalized many code class 
like BCH or Reed-Solomon. In 2015, Robert Gwezheneg expanded Gabidulin 
construction to skew-polynomials including the characteristic zero.

\todo{Présenter ce qu'il y a dans cet article. Le faire à la fin 
pour le détail.}

\section{Ore polynomials}

Throughout this article, we use the following notation: $K$ is a field, 
$\theta : K \to K$ be a ring homomorphism and $\partial : K \to K$ be a 
$\theta$-derivation, i.e. an additive mapping such that $\partial(ab) = 
\theta(a)\partial(b) + \partial(a)b$ for all $a,b \in K$.

We shall denote by $F$ the subfield of $K$ consisting of elements
$a$ such that $\theta(a) = a$ and $\partial(a) = 0$. 
\textbf{We will always assume that the extension $K/F$ is finite.}
This implies in particular that $\theta$ has finite order and thus
is bijective.

%Ore polynomials are a non-commutative generalization of classical 
%polynomials. They are usefull in semi-linear algebra and in the 
%resolution of differential equation. They intervene in the same way as 
%classical polynomials in linear algebra, like endomorphism polynomial or 
%characteristic polynomial. Let's start with a definition

\begin{definition}[Ore polynomial ring]
The ring of Ore polynomials $K[X; \theta, \partial]$ is the ring 
whose elements are polynomials in $X$ over $A$ endowed with the usual 
addition and with the multiplication defined by the rule:
$$X \times a = \theta(a)X + \partial(a), \quad \forall a \in A.$$
%
%An element of this ring is called Ore polynomial.
%
%When $\partial = 0$, Ore ring is called skew polynomial ring and when 
%$A$ is a finite field and $\theta$ the Frobenius morphism, Ore ring is 
%called linearized polynomial ring.
\end{definition}

The notion of degree extends \emph{verbatim} to Ore polynomials: if $P = 
\sum a_iX^i$ is an Ore polynomial, its degree is the largest integer $i$ 
for which $a_i \neq 0$.
Besides, one can prove the existence of a right Euclidean division for 
Ore polynomials: if $A, B \in K[X;\theta,\partial]$ with $B \neq 0$, 
there exist unique $Q, R \in K[X;\theta,\partial]$ with $A = QB+R$ and 
$\deg R < \deg B$. This has the usual consequences: the noncommutative
ring $K[X;\theta,\partial]$ is left-principal, right \textsc{gcd}s and
left \textsc{lcm}s are well defined and can be computed by Euclidean
algorithm. 
Similarly, left Euclidean divisions, left \textsc{gcd}s and right 
\textsc{lcm}s do exist (since our general assumptions imply that
$\theta$ is bijective).

\medskip

\noindent
\textit{Notation:}
In what follows, we will denote by $A \mod B$ the remainder in the 
right division of $A$ by $B$.

\subsubsection*{Pseudo-linear morphisms.}

Another important notion is that of pseudo-linear morphisms. It is 
defined as follows:

\begin{definition}[Pseudo-linear morphism]
Let $M$ and $N$ be two vector spaces over $K$.
A \emph{pseudo-linear morphism} $u : M\to N$ is a map verifying 
$u(ax) = \theta(a)u(x) + \partial(a)x$ for all $a \in K$ and $x \in M$.
\end{definition}

We observe that any pseudo-linear morphism is \emph{a fortiori}
$F$-linear (where $F$ is defined at the beginning of this section).

Pseudo-linear morphisms are relevant in the context of Ore polynomials 
because the Ore multiplication reflects the composition rule of 
pseudo-linear morphisms. More precisely, given a pseudo-linear 
endomorphism $u : M \to M$ and a Ore polynomial $P = \sum_i a_i X^i \in 
K[X;\theta,\partial]$, one defines $P(u) = \sum_i a_i u^i$. One then 
easily checks that $P(u) \circ Q(u) = (PQ)(u)$ where the multiplication 
on the right hand size is the Ore multiplication. In other words, 
denoting by $\End_F(M)$ the ring of $F$-linear maps from $M$ to itself, 
the ``evaluation'' mapping
$$\ev{u} : \quad K[X;\theta,\partial] \to \End_F(M), \quad
P(X) \mapsto P(u)$$
is a ring homomorphism for any pseudo-linear endomorphism $u$.

The case where $M$ is $K$ itself deserves particular attention.
Indeed, we first observe that evaluation is then closely related to
Euclidean division thanks to the formula:
$$\textstyle \ev{u}(P)(a) = 
a \cdot P \mod \big(X - \frac{u(a)}{a}\big)$$
for any pseudo-linear endomorphism $u$ of $K$, any $P \in K[X;\theta,
\partial]$ and any $a \in K$. Second, we have a complete classification
of pseudo-linear endomorphisms of $K$.

\begin{proposition}
The pseudo-linear endomorphisms of $K$ are exactly the maps of
the form $\partial + c\theta$ with $c \in K$.
\end{proposition}

\begin{proof}
It is easily checked that $\partial + c\theta$ is pseudo linear
for all $c\in A$. Conversely, let $u$ be a pseudo-linear morphism.
Set $g = u - \partial$. It is easily checked that $g$ is 
$\theta$-semi linear, i.e. $g(ab) = \theta(a) g(b)$ for all $a, b 
\in K$. Applying this with $b = 1$ and letting $c = g(1)$, we end
up with $g = c \theta$ and so $u = \partial + c\theta$.
\end{proof}

In what follows, we will often use the notation $\ev c$ in place of 
$\ev{\partial + c \theta}$. The properties of $\ev c$ are summarized
in the next proposition.

\begin{proposition}
\label{prop:evc}
\noindent
\textit{1.}
For all $c \in K$, the ring homomorphism $\ev{c}$ is surjective
and its kernel is a principal ideal generated by a polynomial of
degree $[K:F]$ with coefficients in $F$.

\noindent
\textit{2.}
For $c_1, c_2 \in K$, the equality $\ker \ev{c_1} = \ker \ev{c_2}$ holds 
if and only if there exists $a \in K$, $a \neq 0$ such that 
\begin{equation}
\label{eq:}
\ev{c_1}(P) = m_a^{-1} \circ \ev{c_2}(P) \circ m_a
\end{equation}
for all $P \in K[X;\theta,\partial]$, where $m_a : K \to K$ is the
multiplication by $a$.
\end{proposition}

When $\ker \ev{c_1} = \ker \ev{c_2}$, we shall say that $c_1$
and $c_2$ are \emph{equivalent}. Evaluating~\eqref{eq:} at $1$,
we find that $c_1$ and $c_2$ are equivalent if and only if there
exists $a \in K$, $a \neq 0$ such that
$c_1 a = c_2 \theta(a) + \partial(a)$. In particular, we notice 
that the equivalence class of $c \in K$ is exactly the image of 
$x \mapsto \frac{(\partial + c\theta)(x)} x$.

Moreover, when $\theta = \id$, the condition resumes to the equality 
$c_1 = c_2 + \frac{\partial(a)} a$, i.e. to the fact that $c_1 - c_2$ is 
a logarithmic derivative. On the other hand, when $\partial = 0$, the 
condition is equivalent to $c_1 = c_2 \cdot \frac{\theta(a)}a$, that is 
to the fact that $c_1$ and $c_2$ has the same norm over $F$.



\section{Generalized Gabidulin codes}

\begin{definition}[Gabidulin Code]
If $[F:K] = m$ and let $n, k \in \mathbb{N}$ such as $k \leqslant n \leqslant m$. Let $g=(g_1, \dots, g_n) \in K^n$ such as $g_i$ are linearly independant over $F$ and let $c \in K$. Gabidulin code is the set of words : 
$$Gab_c[n,k,g] =\{ ev_c(f)(g_1), \dots, ev_c(f)(g_n)\ | \ f \in K[X, \theta, \partial] \textbf{ and } \deg(f)<k\},$$
it's a code of length $n$ and dimension $k$.
\end{definition}

\begin{definition}[Rank weight]
Let $a = (a_1, \dots, a_n) \in K^n$, the rank weight of $a$ is the dimension of the $F$-vector space generated by $a$.
\end{definition}

\begin{proposition}
Gabidulin codes are Maximal Rank Distance code (MRD code) i.e. the minimum rank distance $d = n-k+1$.
\end{proposition}

\begin{proof}
Let $Gab_c[n,k,g]$ a Gabidulin code and let $a = (a_1, \dots, a_n) \in Gab_c[n,k,g]$ a non-zero word as rank weight $w$. So, there is a Ore polynomial $P$ as $\deg(P) = w$ and $ev_c(P)(a_i) = 0$ for all $i \in \{1, \dots, n\}$. However, $a$ is a word of Gabidulin Code, so there is a Ore polynomial $Q$ as $\deg(Q) \leqslant k-1$ and $a_i = ev_c(Q)(g_i)$. We known that $ev_c(PQ)(g_i) = ev_c(P)(a_i) = 0$ and thus the Ore polynomial PQ is an annihilator polynomial of $n$ elements linearly independant over $F$. So, we have $\deg(PQ) \geqslant n$. Moreover, by construction we have $\deg(PQ) = w + k -1$, so we have $n \leqslant w+k-1$ i.e. $w \geqslant n-k+1$. The equality is given by Singleton bound.
\end{proof}


To decode Gabidulin codes, we have a generalized Gao algorithm. Let $Gab_c[n,k,g]$ a gabidulin code and $b = \{b_1, \dots, b_n\} \in K^n$ a received vector. To decode $b$ we proceed as follows.

\begin{itemize}
\item Step 0, Annihilator : Find the unique Ore polynomial $G_0$ as $ev_c(G_0)(g_i) = 0$ for all $i \in \{1, \dots, n\}$.
\item Step 1, Interpolation : Find the unique Ore polynomial $G_1$ as $ev_c(G_1)(g_i) = b_i$ and $\deg(G_1) \leqslant n-1$.
\item Step 2, Partial gcd : Apply the extend Euclidian algorithm to $G_0$ and $G_1$. Stop when the remainder $G$ has degree $< \frac{n+k}{2}$. If we have to this step $uG_0 + vG_1 = G$.
\item Step 3, Long left division : Apply a left division to $G$ by $v$, say $G = vf_1 + r$. If $r =0$ and $\deg(f_1) < k$ then the word is $f_1$ else, that more than $\frac{d-1}{2}$ errors occurred. 
\end{itemize}


\todo{\begin{itemize}
%\item définition, longueur, dimension
%\item distance rang (ou Hamming-rang selon le cas)
%\item calcul de la distance minimale
\item Liens entre les def/prop
\item algorithme de décodage de Gao
\item donner des exemples : 1) sur les corps finis et 2) avec des dérivations
\end{itemize}}

\section{Implementation}

\todo{\begin{itemize}
\item dire ce qu'on a implémenté
\item donner quelques benchmarks
\end{itemize}}

\end{document}
