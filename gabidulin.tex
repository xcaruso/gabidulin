\documentclass[a4paper]{llncs}

\usepackage[utf8]{inputenc}
\usepackage{amsmath}
\usepackage{amsfonts}
\usepackage{amssymb}
\usepackage{enumerate}
\usepackage{color}
\usepackage[english]{babel}
\usepackage{microtype}

\definecolor{todo}{rgb}{0.9,0,0}

\author{Caruso Xavier \and Durand Amaury}
\institute{CNRS, Institut Mathématique de Bordeaux\\
Rennes ? Versailles ? Besançon ?}

\def\todo#1{{\color{todo} #1}}

\title{Reed--Solomon--Gabidulin Codes}

\newcommand{\ZZ}{\mathbb Z}
\newcommand{\QQ}{\mathbb Q}
\newcommand{\FF}{\mathbb F}

\newcommand{\id}{\textrm{\rm id}}
\newcommand{\End}{\textrm{\rm End}}
\newcommand{\GL}{\textrm{\rm GL}}
\newcommand{\Frob}{\textrm{\rm Frob}}
\newcommand{\ev}[1]{\textrm{\rm ev}_{#1}}
\renewcommand{\mod}{\,\%\,}

\newcommand{\rgcd}{\textsc{rgcd}}
\newcommand{\llcm}{\textsc{llcm}}

\newcommand{\bc}{\textbf{c}}
\newcommand{\bg}{\textbf{g}}
\newcommand{\RSG}{\textrm{\rm RSG}}
\newcommand{\wH}{w_{\textrm{\rm H}}}
\newcommand{\wrH}{w_{\textrm{\rm rH}}}
\newcommand{\drH}{d_{\textrm{\rm rH}}}
\newcommand{\good}{\textrm{\rm good}}

\begin{document}
\maketitle

\section{Introduction}

In 1985, Gabidulin Ernst introduced a Reed-Solomon-like code 
construction in rank metric. This Gabidulin code class use linearised 
polynomial concept. In 2013, Wachter-Zeh Antonia proposed efficient 
implementations of operations with linearized polynomials as well as an 
equivalent of Gao's algorithm for decoding Gabidulin code. Following 
years, Boucher Delphine and Ulmer Felix have worked on skew coding 
theory using Ore's polynomials and they have generalized many code class 
like BCH or Reed-Solomon. In 2015, Robert Gwezheneg expanded Gabidulin 
construction to skew-polynomials including the characteristic zero.

\todo{Présenter ce qu'il y a dans cet article. Le faire à la fin 
pour le détail.}

\section{Ore polynomials}

Throughout this article, we use the following notation: $K$ is a field, 
$\theta : K \to K$ be a ring homomorphism and $\partial : K \to K$ be a 
$\theta$-derivation, i.e. an additive mapping such that $\partial(ab) = 
\theta(a)\partial(b) + \partial(a)b$ for all $a,b \in K$.

We shall denote by $F$ the subfield of $K$ consisting of elements
$a$ such that $\theta(a) = a$ and $\partial(a) = 0$. 
\textbf{We will always assume that the extension $K/F$ is finite}
and will denote by $r$ its degree. Our assumption implies in particular 
that $\theta$ has finite order and thus is bijective.

%Ore polynomials are a non-commutative generalization of classical 
%polynomials. They are usefull in semi-linear algebra and in the 
%resolution of differential equation. They intervene in the same way as 
%classical polynomials in linear algebra, like endomorphism polynomial or 
%characteristic polynomial. Let's start with a definition

\begin{definition}[Ore polynomial ring]
The ring of Ore polynomials $K[X; \theta, \partial]$ is the ring 
whose elements are polynomials in $X$ over $A$ endowed with the usual 
addition and with the multiplication defined by the rule:
$$X \times a = \theta(a)X + \partial(a), \quad \forall a \in A.$$
%
%An element of this ring is called Ore polynomial.
%
%When $\partial = 0$, Ore ring is called skew polynomial ring and when 
%$A$ is a finite field and $\theta$ the Frobenius morphism, Ore ring is 
%called linearized polynomial ring.
\end{definition}

\begin{example}
\label{ex:Ore}
Throughout this article, we will illustrate all our constructions 
with two following examples:
\begin{itemize}
\renewcommand{\itemsep}{1ex}
\item[(1)] (This setting is the one in which Gabidulin codes were
first defined by Gabidulin in \cite{gabidulin}, with a slightly different
vocabulary.)
Let $p$ be a prime number, $q$ be a power of $p$ and $r$
be a positive integer. We let $\FF_{q^r}$ denote a finite field
with cardinality $q^m$. We endow it with the Frobenius $\Frob_q :
x \mapsto x^q$. The first Ore ring we will be interested in is
$\FF_{q^r}[X ; \Frob_q, 0]$. In this setting, the subfield $F$ of 
$K = \FF_{q^r}$ we have introduced is $\FF_q$. The degree of the
extension $K/F$ is then $r$.
\item[(1')] More generally, one can pick an arbitrary field $K$,
endow it with a finite order automorphism $\theta$ and consider the
Ore ring $K[X,\theta,0]$. Beyond the case of finite fields, natural
examples are cyclotomic extensions of $\QQ$ or Kummer extensions.
This case was addressed in Robert's thesis \cite{robert}.
\item[(2)] Let $k$ be a field of characteristic $p$. We consider the field 
$K = k(t)$ and endow it with the natural derivation $\frac d{dt}$. We can 
then form the Ore ring $k(t)[X,\id,\frac d{dt}]$. Here the subfield $F$ 
of $K$ is $k(t^p)$ and the degree of the extension $K/F$ is then $p$.
\end{itemize}
\end{example}

%\begin{example}
%Let $K = GF_7(t)$ the fraction field of univariate polynomial on the 
%field at 7 elements. Let $\partial = \frac{\partial}{\partial t}$ a 
%classic polynomial derivation. In this article, we will interesting in 
%this red thread example : $K[X, \id, \partial] = GF_7(t)[X, \id, 
%\frac{\partial}{\partial t}]$, and $F$ is the subfield $F = \{x \in 
%GF_7(t) \ | \ \frac{\partial x}{\partial t} = 0\} = GF_7(t^7)$.
%\end{example}

The notion of degree extends \emph{verbatim} to Ore polynomials: if $P = 
\sum a_iX^i$ is an Ore polynomial, its degree is the largest integer $i$ 
for which $a_i \neq 0$.
Besides, one can prove the existence of a right Euclidean division for 
Ore polynomials: if $A, B \in K[X;\theta,\partial]$ with $B \neq 0$, 
there exist unique $Q, R \in K[X;\theta,\partial]$ with $A = QB+R$ and 
$\deg R < \deg B$. This has the usual consequences: the noncommutative
ring $K[X;\theta,\partial]$ is left-principal, right \textsc{gcd}s and
left \textsc{lcm}s are well defined and can be computed by Euclidean
algorithm. 
Similarly, left Euclidean divisions, left \textsc{gcd}s and right 
\textsc{lcm}s do exist (since our general assumptions imply that
$\theta$ is bijective).

\medskip

\noindent
\textit{Notation:}
In what follows, we will denote by $A \mod B$ the remainder in the 
right division of $A$ by $B$.

\subsubsection*{The centre.}

Recall that the centre of a noncommutative ring $A$ is by definition
the subset of $A$ consisting of elements $x$ such that $xy = yx$ for
all $y \in A$. We observe in particular that the centre of $A$ is a
commutative subring of $A$.
In the case of Ore polynomials, the centre can actually be computed
precisely \cite{}. 
In what follows, we will not need a complete description but only
the general structure of the centre as given by the next proposition.

\begin{proposition}
\label{prop:centre}
There exists a central Ore polynomial $Z(X) \in K[X; \theta, \partial]$ of 
degree $r$ such that the centre of $K[X; \theta, \partial]$ is 
$F[Z(X)]$, \emph{i.e.} the subset of Ore polynomials that can be
written as a polynomial in $Z(X)$ with coefficient in $F$.
\end{proposition}

We observe that the equality:
$$a_0 + a_1 Z(X) + \cdots + a_d Z(X)^d 
= b_0 + b_1 Z(X) + \cdots + a_e Z(X)^e$$
implies readily that $d = e$ (compare the degrees) and $a_i = b_i$ 
for all $i$. As a consequence the centre $F[Z(X)]$ is an actual
(commutative) ring of univariate polynomials with coefficients in $F$.

On the other hand, we draw the attention of the reader to the fact that 
the properties of Proposition~\ref{prop:centre} do not determine $Z(X)$ 
uniquely but only up to an additive constant in $F$.

\begin{example}
\label{ex:centre}
We continue Example \ref{ex:Ore}. In the settings~(1) and~(1'), it is 
easily seen that the centre of $K[X;\theta,0]$ is $F[X^r]$.
In the setting~(2), the centre of $k(t)[X; \id, \frac d{dt}]$ (where
$k$ is a field of characteristic $p$) is $k(t^p)[X^p]$. Let us prove
the latter assertion. First we observe that $t^p$ and $X^p$ are 
central elements. Indeed $t^p$ is central because its derivative
vanishes. On the other hand, for all $f \in k(t)$, iterating the 
Leibniz relation, we get:
$$f X^p = \sum_{i=0}^p \binom i p \frac{d^i f}{dt^i} X^{p-i}
= f X^p + \frac{d^p f}{dt^p}$$
and we check that the last term $\frac{d^p f}{dt^p}$ vanishes.

\noindent
Conversely, let $P(X) = f_0 + f_1 X + \cdots + f_d X^d \in k(t)[X; 
\id, \frac d{dt}]$ be a central element. From $X P(X) = P(X) X$, we
derive $\frac {df_i}{dt} = 0$ for all $i$, for what it follows $f_i 
\in k(t^p)$ for all $i$. Up to subtracting to $P(X)$ a central element,
we can assume without loss of generality that $a_i = 0$ as soon as 
$p$ divides $i$. Comparing the coefficients in $X^{d-1}$
in $t P(X)$ and $P(X) t$, we obtain $d f_d = 0$. Thus $f_d = 0$
(whether or not $p$ divides $d$).
Repeating the argument again and again, we find $P(X) = 0$ as 
expected.
\end{example}

\subsubsection*{Pseudo-linear morphisms.}

Another important notion is that of pseudo-linear morphisms. It is 
defined as follows:

\begin{definition}[Pseudo-linear morphism]
Let $M$ and $N$ be two vector spaces over $K$.
A \emph{pseudo-linear morphism} $u : M\to N$ is a map verifying 
$u(ax) = \theta(a)u(x) + \partial(a)x$ for all $a \in K$ and $x \in M$.
\end{definition}

We observe that any pseudo-linear morphism is \emph{a fortiori}
$F$-linear (where $F$ is defined at the beginning of this section).

Pseudo-linear morphisms are relevant in the context of Ore polynomials 
because the Ore multiplication reflects the composition rule of 
pseudo-linear morphisms. More precisely, given a pseudo-linear 
endomorphism $u : M \to M$ and a Ore polynomial $P = \sum_i a_i X^i \in 
K[X;\theta,\partial]$, one defines $P(u) = \sum_i a_i u^i$. One then 
easily checks that $P(u) \circ Q(u) = (PQ)(u)$ where the multiplication 
on the right hand size is the Ore multiplication. In other words, 
denoting by $\End_F(M)$ the ring of $F$-linear maps from $M$ to itself, 
the ``evaluation'' mapping
$$\ev{u} : \quad K[X;\theta,\partial] \to \End_F(M), \quad
P(X) \mapsto P(u)$$
is a ring homomorphism for any pseudo-linear endomorphism $u$.

The case where $M$ is $K$ itself deserves particular attention.
Indeed, we first observe that evaluation is then closely related to
Euclidean division thanks to the formula:
\begin{equation}
\label{eq:evanddiv}
\textstyle \ev{u}(P)(a) = 
a \cdot P \mod \big(X - \frac{u(a)}{a}\big)
\end{equation}
for any pseudo-linear endomorphism $u$ of $K$, any $P \in K[X;\theta,
\partial]$ and any $a \in K$. Second, we have a complete classification
of pseudo-linear endomorphisms of $K$.

\begin{proposition}
The pseudo-linear endomorphisms of $K$ are exactly the maps of
the form $\partial + c\theta$ with $c \in K$.
\end{proposition}

\begin{proof}
It is easily checked that $\partial + c\theta$ is pseudo linear
for all $c\in A$. Conversely, let $u$ be a pseudo-linear morphism.
Set $g = u - \partial$. It is easily checked that $g$ is 
$\theta$-semi linear, i.e. $g(ab) = \theta(a) g(b)$ for all $a, b 
\in K$. Applying this with $b = 1$ and letting $c = g(1)$, we end
up with $g = c \theta$ and so $u = \partial + c\theta$. \qed
\end{proof}

In what follows, we will often use the notation $\ev c$ in place of 
$\ev{\partial + c \theta}$.

\paragraph{Main properties of the $\ev c$'s.}

We denote by $K_\good$ the subset of $K$ consisting of elements $c$ for 
which $\partial + c\theta$ is not of the form $a{\cdot}\id$ with $a \in 
F$.

\begin{proposition}
\label{prop:evc}
For all $c \in K_\good$, the ring homomorphism $\ev{c}$ is surjective
and its kernel is a principal ideal generated by $Z(X) - N(c)$
for some element $N(c) \in F$.
\end{proposition}

\begin{remark}
The function $N$ defined by Proposition \ref{prop:evc} above is not 
canonical since it depends on the choice of the constant coefficient 
of $Z(X)$. Two different choices lead to functions $N$ and $N'$ such
that $N' = N + a$ for some constant $a \in F$.
\end{remark}

%\begin{example}
%For $c = 0 \in GF_7$, $\ker (\ev 0)$ is generated by the polynomial $X^7$. Indeed, we have $\ev 0 (X^7) = (\frac{\partial}{\partial t})^7$ and so, for all $P \in F = GF_7(t^7)$, $\ev 0 (P) = 0$.
%\end{example}

\begin{definition}
\label{def:equiv}
Let $c_1, c_2 \in K_\good$.
We say that $c_1$ and $c_2$ are \emph{equivalent} if
$\ker \ev{c_1} = \ker \ev{c_2}$ or, equivalently, $N(c_1) = N(c_2)$.
\end{definition}

\noindent
Using Noether--Skolem Theorem, one can prove the following
characterization:

\begin{lemma}
\label{lem:equiv}
The elements $c_1$ and $c_2$ are equivalent if and only if there exists 
$a \in K$, $a \neq 0$ such that $c_1 a = c_2 \theta(a) + \partial(a)$.

\noindent
In particular, the equivalence class of $c \in K$ is exactly the image 
of $x \mapsto \frac{(\partial + c\theta)(x)} x$.
\end{lemma}

\begin{example}
\label{ex:equiv}
Let us first focus on the settings~(1) and~(1') of Example~\ref{ex:Ore}. 
The subset $K_\good$ is then $K \backslash \{0\}$. Moreover if we have 
chosen $Z(X) = X^r$ (see Example~\ref{ex:centre}), it is not difficult
to prove that the map $N$ is the norm of $K$ over $F$.
In this context, the characterization of Lemma~\ref{lem:equiv} is 
a classical consequence of Hilbert 90 theorem which says that an
element has norm $1$ if and only if it can be written 
$\frac{\theta(a)} a$ for some $a \neq 0$.

\noindent
When $K = \FF_{q^m}$ and $\theta = \Frob_q$, we have 
$N(c) = c^{1 + q + q^2 + \cdots + q^{m-1}}$. In this case, the image of 
$N$ is $\FF_q^\star$ and there is exactly $q{-}1$ equivalence classes 
for the equivalence relation introduced in Definition~\ref{def:equiv}.

In the setting~(2), we have $K_\good = K$. Moreover, with the 
normalization $Z(X) = X^p$, one can prove\footnote{Through the proof is 
not obvious.} that $N(f) = \frac {d^{p-1}f}{dt^{p-1}} + f^p$ for any $f 
\in k(t)$. Here, Lemma~\ref{lem:equiv} asserts that $N(f) = N(g)$ if
and only if the difference $f-g$ is a logarithmic derivative.
It is easily seen that a polynomial cannot be a logarithmic derivative.
Consequently the elements of $k[t]$ are pairwise nonequivalent,
implying in particular that there are infinitely many equivalence
classes for this relation.
\end{example}

\begin{proposition}
\label{prop:evbc}
Let $\bc = (c_1, \ldots, c_m)$ be a tuple of $m$ pairwise non-equivalent 
elements of $K_\good$.
Then the ring homomorphism:
$$\ev \bc : \quad K[X;\theta,\partial] \to \End_F(M)^m, \quad
P \mapsto (\ev{c_1}(P), \ldots, \ev{c_m}(P))$$
is surjective and its kernel is the ideal generated by
$\prod_{i=1}^m (Z(X) - N(c_i))$.
\end{proposition}

\begin{proof}
Let $P \in \ker \ev\bc$. From the fact that $\ev{c_i}(P)$ vanishes,
we deduce that $P$ is a multiple of $Z(X) - N(c_i)$. We know moreover
by assumption that the $N(c_i)$'s are pairwise distinct. It follows
from this that:
$$\llcm(Z(X)-N(c_1), \ldots, Z(X)-N(c_m)) = 
\prod_{i=1}^m (Z(X) - N(c_i)).$$
Therefore $\prod_{i=1}^m (Z(X) - N(c_i))$ must divide $P$ and the
assertion about the kernel is proved. The surjectivity of $\ev\bc$
follows by comparing dimensions.
\end{proof}

\section{Reed--Solomon--Gabidulin codes}

\todo{Écrire une introduction}

We keep the notations of the previous section. In particular, we recall 
that $K_\good$ is the subset of $K$ consisting of elements $c$ for which 
$\partial + c\theta$ is not of the form $a{\cdot}\id$ with $a \in F$.

\subsubsection*{Setting.}

Throughout this section, we fix a positive integer $m$. We consider a 
family $\bc = (c_1, \ldots, c_m)$ of $m$ elements of $K_\good$ which are 
pairwise non-equivalent in the sense of Definition \ref{def:equiv}.
Moreover, for each $i \in \{1,\ldots,m\}$, we pick a positive integer
$n_i$ together with a family $\bg_i = (g_{i,1}, \ldots, g_{i,n_i})$ of 
$F$-linearly independant elements of $K$. The latter condition obviously
implies that $n_i \leq [K:F]$ for all $i$.
We set $n = n_1 + \ldots + n_m$ and we fix an extra positive integer
$k < n$.
To all these data, we associate the $F$-linear mapping:
$$\begin{array}{rcl}
\gamma_{k,\bc,\bg} : \, K[X;\theta,\partial]_{<k} & \longrightarrow 
 & K^{n_1} \times K^{n_2} \times \cdots \times K^{n_m} \smallskip \\
P(X) & \mapsto 
 & \big(\ev{c_1}(P)(g_{1,1}), \ev{c_1}(P)(g_{1,2}), \ldots, \ev{c_1}(P)(g_{1,n_1}), \\
&& \phantom{\big(}\ev{c_2}(P)(g_{2,1}), \ev{c_2}(P)(g_{2,2}), \ldots, \ev{c_2}(P)(g_{2,n_2}), \\
&& \phantom{\big(}\ldots, \\
&& \phantom{\big(}\ev{c_m}(P)(g_{m,1}), \ev{c_m}(P)(g_{m,2}), \ldots, \ev{c_m}(P)(g_{m,n_m})\big)
\end{array}$$
Thanks to Eq.~\eqref{eq:evanddiv}, the mapping $\gamma_{k,\bc,\bg}$
can be rewritten in terms of Euclidean divisions. More precisely,
for $1 \leq i \leq m$ and $1 \leq j \leq n_i$, letting:
\begin{equation}
\label{eq:aij}
a_{i,j} = \frac{(\partial + c_i\theta)(g_{i,j})}{g_{i,j}}
\end{equation}
we have $\ev{c_i}(g_{i,j}) = g_{i,j} \cdot P \mod (X - a_{i,j})$.
It is worth remarking that the assumption we made on the $c_i$'s
and $g_{i,j}$'s can be translated easily on the $a_{i,j}$, as shown
by the next lemma.

\begin{lemma}
\label{lem:llcmaij}
With the above notations, the Ore polynomial
$$L = \llcm((X - a_{i,j})_{1 \leq i \leq m,\,1 \leq j \leq n_i})$$
has degree $n$.
\end{lemma}

\begin{proof}
Assume by contradiction that $\deg L < n$.
For each $i \in \{1,\ldots,m\}$, pick some additional $g_{i,j}$'s
for $n_i < j \leq r$ in order to form a basis $(g_{i,j})_{1 \leq j 
\leq r}$ of $K$ over $F$. We define the corresponding $a_{i,j}$'s
accordingly using \eqref{eq:aij}. 
Set $L' = \llcm(X - a_{i,j})_{1 \leq i \leq m,\,1 \leq j \leq r})$.
From
$L' = \llcm(L, (X - a_{i,j})_{1 \leq i \leq m,\,n_i < j \leq r}$,
we obtain 
$$\deg L' \leq \deg L + \sum_{i=1}^m (r - n_i) < rm.$$
Moreover, for any given $i \in \{1,\ldots, m\}$, the linear mapping 
$\ev{c_i}(L')$ vanishes on $g_{i,j}$ for all $j \in \{1, \ldots,r\}$. 
Since they form a basis of $K$ over $F$, we get $\ev{c_i}(L') = 0$.
Therefore $L'$ lies in the kernel of the ring homomorphism $\ev\bc =
(\ev{c_1}, \ldots, \ev{c_m})$. By Proposition \ref{prop:evbc},
$L'$ has to be a multiple of $\prod_{i=1}^m (Z(X) - N(c_i))$,
which is a contradiction since the latter polynomial has degree $rm > 
\deg L'$.
\end{proof}

\begin{example}
\label{ex:aij}
Consider first the setting~(1) of Example~\ref{ex:Ore}.
Let $g$ be a multiplicative generator of $\FF_{q^r}^\star$. Its norm
over $\FF_q$ is a multiplicative generator of $\FF_q^\star$. By what
we did in Example~\ref{ex:equiv}, the elements $c_i = g^i$ for $0 
\leq i < q-1$ are pairwise nonequivalent. Moreover $(1, g, \ldots,
g^{r-1})$ is a basis of $\FF_{q^m}$ over $\FF_q$. One can then
take $n_i = r$ for all $i$ and $g_{i,j} = g^j$. With these parameters,
we easily compute $a_{i,j} = c_i \cdot \Frob_q(g_{i,j}) \cdot g_{i,j}^{-1}
= g^{i + (q-1)j}$. 
%The set of all $a_{i,j}$'s is then the set of 
%$g^e$'s for $0 \leq e < (q{-}1)r$.

We now move to the setting~(2). By Example~\ref{ex:equiv} again,
we can take any family $(c_1, \ldots, c_m)$ of pairwise distinct
polynomials. Moreover a basis of $k(t^p)$ over $k(t)$ is obviously
$(1, t, \ldots, t^{p-1})$. Therefore, we can take $n_i = p$ and
$g_{i,j} = t^{j-1}$ for $1 \leq j \leq p$. A direct computation
leads to $a_{i,j} = \frac{j-1} t + c_i$.
\end{example}

\subsubsection*{Definition and first properties.}

We are now ready to define Gabidulin codes in the extended framework
discussed in the introduction of this section.

\begin{definition}
With the previous notations, the \emph{Reed--Solomon--Gabidulin (RSG for 
short) code} $\RSG_{k,\bc,\bg}$ associated to $\bc$ and $\bg$ is the 
image of $\gamma_{k,\bc,\bg}$. \end{definition}

\begin{example}
To complete the red thread example, we'll considere the Gabidulin Code : $Gab_0[7,3,g]$ where $g = (t^i)$ for $i \in \{0, \dots, 6\}$. It is of length  $7$ anf dimension $3$.
\end{example}

It is well known that the relevant distance for Gabidulin codes is not 
the Hamming distance but the rank distance. In the context of Gabidulin 
codes introduced above, we shall need another distance which is a 
mixture between Hamming and rank distance. It is defined as follows.

\begin{definition}
Let $x = (x_{i,j})_{1 \leq i \leq m, \; 1 \leq j \leq n_i} \in
K^{n_1} \times K^{n_2} \times \cdots \times K^{n_m}$.
The \emph{rank-Hamming weight} of $x$ is:
$$\wrH(x) = 
\sum_{i=1}^m \dim_F \big< x_{i,1}, x_{i,2}, \ldots, x_{i,{m_i}} \big>_F.$$
Given $x, y \in K^{n_1} \times K^{n_2} \times \cdots \times K^{n_m}$, 
the {rank-Hamming distance} between $x$ and $y$ is $\drH(x,y) = 
\wrH(x-y)$.
\end{definition}

\begin{remark}
The weight $\wrH$ is finer that the usual Hamming weight in the 
sense that, for all $x \in K^{n_1} \times \cdots \times K^{n_m}$, 
we have $\wrH(x) \leq \wH(x)$ if $\wH$ denotes the Hamming weight.
\end{remark}

The RSG codes we have defined extend the classical notion of Gabidulin 
codes introduced in \cite{gabidulin}. More precisely, the latter correspond to 
the case where $m = 1$, $\partial = 0$ and $K$ is a finite field. 
Relaxing the assumption on $K$, we obtain the generalized Gabidulin 
codes defined by Robert in his thesis \cite{robert}. In particular, in this 
case, the rank-Hamming distance is the usual rank distance.

On the other hand, when $\theta = \id$ and $\partial = 0$ (that is $F = 
K$), the notion of RSG code is nothing but the standard notion of 
Reed--Solomon code and the rank-Hamming distance reduces to the 
usual Hamming distance.

\begin{proposition}
The code $\RSG_{k,\bc,\bg}$ has length $n$,
dimension $k$ and minimal distance $d = n - k + 1$.
\end{proposition}


\begin{proof}
Let $x \in \RSG_{k,\bc,\bg}$ be a nonzero codeword. Let $P \in 
K[X;\theta,\partial]$ with $\deg P < k$ and $\gamma_{k,\bc,\bg}(P) = x$.
Let $E_i$ denote the $F$-span of $g_{i,1}, \ldots, g_{i,n_i}$. By
definition of the rank-Hamming weight, we know that
$\wrH(x) = \sum_{i=1}^m \dim_F \ev{c_i}(P)(E_i)$.
Set $\delta_i = \dim_F \ker \ev{c_i}(P)$. By the rank-nullity
theorem, $\delta_i + \dim_F \ev{c_i}(P)(E_i) \geq n_i$. By summing
up on $i$, we get:
\begin{equation}
\label{eq:sumdeltai}
\sum_{i=1}^m \delta_i \geq n - \wrH(x).
\end{equation}
For $i \in \{1,\ldots,d\}$, let $(h_{i,1}, \ldots, h_{i,\delta_i})$
be a basis of $\ker \ev{c_i}(P)$. 
Set moreover $b_{i,j} = \frac{(\partial + c_i\theta)(h_{i,j})}{b_{i,j}}$
and $L = \llcm((X - b_{i,j})_{1 \leq i \leq m, \, 1 \leq j \leq \delta_i})$.
By copying the proof of Lemma \ref{lem:llcmaij}, we end up with
$\deg L = \sum_{i=1}^m \delta_i$. Furthermore since $\ev{c_i}(P)$
vanishes on all $h_{i,j}$'s, the Ore polynomial $P$ has to be a 
multiple of $L$. Thus $k > \deg P \geq \sum_{i=1}^m \delta_i$.
Comparing with \eqref{eq:sumdeltai}, we find $k > n - \wrH(x)$,
\emph{i.e.} $\wrH(x) \geq n - k + 1$.
We have then proved that the minimal distance of $\RSG_{k,\bc,\bg}$
is at least $n - k + 1$.
The inequality in the other direction, together with the fact that
$\RSG_{k,\bc,\bg}$ has dimension $k$, now follow from Singleton bound.
\end{proof}

\begin{example}
Our example code, $Gab_0[7,3,g]$ have a minimal rank distance $5$ and it can correct an error of weight $2$. 
\end{example}

\subsubsection*{Decoding Reed--Solomon--Gabidulin codes.}

To decode Gabidulin codes, we have a generalized Gao algorithm. Let $Gab_c[n,k,g]$ a gabidulin code and $b = \{b_1, \dots, b_n\} \in K^n$ a received vector. To decode $b$ we proceed as follows.

\begin{itemize}
\item Step 0, Annihilator : Find the unique Ore polynomial $G_0$ as $\ev{c}(G_0)(g_i) = 0$ for all $i \in \{1, \dots, n\}$.
\item Step 1, Interpolation : Find the unique Ore polynomial $G_1$ as $\ev{c}(G_1)(g_i) = b_i$ and $\deg(G_1) \leqslant n-1$.
\item Step 2, Partial gcd : Apply the extend Euclidian algorithm to $G_0$ and $G_1$. Stop when the remainder $G$ has degree $< \frac{n+k}{2}$. If we have to this step $uG_0 + vG_1 = G$.
\item Step 3, Long left division : Apply a left division to $G$ by $v$, say $G = vf_1 + r$. If $r =0$ and $\deg(f_1) < k$ then the word is $f_1$ else, that more than $\frac{d-1}{2}$ errors occurred. 
\end{itemize}


\todo{Ajouter un exemple d'application de l'algorithme.}

\section{Implementation}

\todo{\begin{itemize}
\item dire ce qu'on a implémenté
\item donner quelques benchmarks
\end{itemize}}

\begin{thebibliography}{99}
\bibitem{gabidulin}
  Ernst Gabidulin,
  \emph{Theory of codes with maximum rank distance},
  Problemy Peredachi Informatsii \textbf{21} (1985), no.~1, 3--16.

\bibitem{robert}
  Gwezheneg Robert,
  \emph{Codes de Gabidulin en caractéristique nulle : application au codage espace-temps},
  PhD thesis (2015)

\end{thebibliography}

\end{document}
